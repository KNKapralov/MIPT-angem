\begin{enumerate}
	\item Дайте определение:
    \begin{tasks}(1)
        \task Cкалярное произведение $\vec a$ и $\vec b$:\ \hrulefill\par\hrulefill\par\hrulefill
        \task Векторное произведение $\vec a$ и $\vec b$:\ \hrulefill\par\hrulefill\par\hrulefill
        \task Правая тройка векторов:\ \hrulefill\par\hrulefill\par\hrulefill\par\hrulefill
        \task Ортонормированный базис:\ \hrulefill\par\hrulefill\par\hrulefill
        \task Ортогональная проекция $\vec a$ на прямую с направляющим вектором $\vec b$:\ \hrulefill\par\hrulefill\par\hrulefill
    \end{tasks}
    
    \item В базисе ${\vex e_1, \vec e_2, \vec e_3}$ заданы вектора $\vec a = \alpha_1 \vec e_1 + \alpha_2 \vec e_2 + \alpha_3 \vec e_3$ и $\vec b = \beta_1 \vec e_1 + \beta_2 \vec e_2 + \beta_3 \vec e_3$.
    \begin{tasks}(1)
        \task $(\vec a, \vec b)= $ \ \hrulefill\par\hrulefill\par\hrulefill
        \task $|\vec_a|=$ \ \hrulefill\par\hrulefill\par\hrulefill
        \task $\cos(\angle (\vec a, vec b))=$:\ \hrulefill\par\hrulefill\par\hrulefill\par\hrulefill
        \task если ${\vex e_1, \vec e_2, \vec e_3}$ ОНБ, то $(\vec a, \vec b)=$ :\ \hrulefill\par\hrulefill\par\hrulefill
    \end{tasks}
    
    \item Перечислите известные вам свойства
    	\task скалярного произведения $\vec a$ и $\vec b$:\ \hrulefill\par\hrulefill\par\hrulefill
        \task векторного произведения(в том числе геометрический смысл) $\vec a$ и $\vec b$:\ \hrulefill\par\hrulefill\par\hrulefill
    
    \item В ОНБ ${\vex e_1, \vec e_2, \vec e_3}$ заданы вектора $\vec a = \alpha_1 \vec e_1 + \alpha_2 \vec e_2 + \alpha_3 \vec e_3$ и $\vec b = \beta_1 \vec e_1 + \beta_2 \vec e_2 + \beta_3 \vec e_3$.
    	\task $[\vec e_1, \vec e_1]$  \hspace{28} $[\vec e_1, \vec e_2]$ \hspace{28} $[\vec e_1, \vec e_3]$ 
   		 \task $[\vec a, \vec b]=$ 
\end{enumerate}