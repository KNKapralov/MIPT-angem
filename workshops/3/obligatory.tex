\begin{enumerate}
 \item Даны вектора $\vec a = \vec e_1 + 2 \vec e_2 + 3 \vec e_3$ и $\vec b = 7 \vec e_1 - 2 \vec e_2 -  \vec e_3$, записанные в базисе ${\vec e_1, \vec e_2, \vec e_3 }$. Вычислите:  
   
   \begin{tasks}(3)
      \task скалярное и векторное произведениея $\vec a$ и $\vec b$, если ${\vec e_1, \vec e_2, \vec e_3 }$ - ОНБ.
    \task скалярное произведение $\vec a$ и $\vec b$, если $|\vec e_1|=|\vec e_2|=|\vec e_3|=1$ и $\angle (\vec e_1, \vec e_2)= \angle (\vec e_1, \vec e_3)=90 \circ \ \angle (\vec e_2, \vec e_3) = 60 \circ$
    \task векторное произведение $\vec a$ и $\vec b$, если $|\vec e_1|=|\vec e_2|=|\vec e_3|=1$ и $\angle (\vec e_1, \vec e_2)= \angle (\vec e_1, \vec e_3)= 90 \circ \ \angle (\vec e_2, \vec e_3) = 30 \circ$
   \end{tasks}
   
   \item Найти угол между векторами $\vec a = 3 \vec e_1 + 4 \vec e_3$ и $\vec b = 4 \vec e_1 + 3 \vec e_2$, записанными в базисе ${\vec e_1, \vec e_2, \vec e_3 }$
    \begin{tasks}(3)
    	\task ${\vec e_1, \vec e_2, \vec e_3 }$ - ОНБ
    	\task $|\vec e_1|=|\vec e_2|=|\vec e_3|=1$ и $\angle (\vec e_1, \vec e_2)= \angle (\vec e_2, \vec e_3)= \angle (\vec e_1, \vec e_3) = 60 \circ$
     \end{tasks}
    
	    
	    
	\item Найти ортогональную проекцию вектора $\vec a = \vec e_1 + 2 \vec e_2 + 3 \vec e_3$ на прямую с направляющим вектором $\vec b = 3 \vec e_1 + 4 \vec e_3$, если 
		\begin{tasks}(3)
	       \task ${\vec e_1, \vec e_2, \vec e_3 }$ - ОНБ
	       \task $|\vec e_1|=|\vec e_2|=|\vec e_3|=1$ и $\angle (\vec e_1, \vec e_2)= \angle (\vec e_2, \vec e_3)= \angle (\vec e_1, \vec e_3) = 60 \circ$
	    \end{tasks}
	    
	\item Даны векторы  $\vec a = \vec e_1 +  \vec e_2 +  \vec e_3$ и $\vec b = \vec e_1 +  \vec e_3$, записанные в базисе ${\vec e_1, \vec e_2, \vec e_3 }$, где $|\vec e_1|=|\vec e_2|=|\vec e_3|=2$ и $\angle (\vec e_1, \vec e_2)= \angle (\vec e_2, \vec e_3)= \angle (\vec e_1, \vec e_3) = 30 \circ$
		\begin{tasks}(3)
	       \task Для произвольные $\vec a$ и $\vec b$. Упростить выражения: 
	       	$$[\vec a + \vec b ,\vec a - \vec b] \hspace{28} (\vec a + \vec b, \vec a - \vec b)$$   
	       \task Вычислить упрощенные выражения в случае $\vec a = \vec e_1 +  \vec e_2 +  \vec e_3$ и $\vec b = \vec e_1 +  \vec e_3$, где  ${\vec e_1, \vec e_2, \vec e_3 }$ - базис , $|\vec e_1|=|\vec e_2|=|\vec e_3|=2$ и $\angle (\vec e_1, \vec e_2)= \angle (\vec e_2, \vec e_3)= \angle (\vec e_1, \vec e_3) = 30 \circ$
	    \end{tasks}
	
	 
 	\item В $\triangle ABC$ проведена биссектриса $AD$. Найти координаты вектора $AD$ в базисе, образованном векторами $\overrightarrow{AB}$ и $\overrightarrow{AC}$.

	 
	 \item Вычислить $[\vec a,[\vec b, \vec c]]$, если 
	$$\vec a = \begin{pmatrix}
	    	1 \\ 2 \\ 3
	    \end{pmatrix}, \hspace{28}
	   \vec b = \begin{pmatrix}
	    	3 \\ 2 \\ 1	    
	    \end{pmatrix},\hspace{28}
	    \vec c = \begin{pmatrix}
	    	1 \\ 1 \\ 1  
	    \end{pmatrix} \\$$
	    Базис считать ортонормированным. 
\end{enumerate}