% Common
\def \CommonOne {%
    \item Пусть даны $N$ точек $A_1, A_2, \dots, A_n$. Найти \textbf{все} точки $O$ такие, что
    $$\overrightarrow{OA_1} + \overrightarrow{OA_2} + \dots + \overrightarrow{OA_n} = \overrightarrow{0}.$$
}

\def \CommonTwo {%
	\item Докажите, что отрезок, соединяющий середины диагонали трапеции, параллелен основаниям и равен их полусумме.}
% simple

\def \SimpleOne {%
	\item
	Три точки
	$$A(x, y), \quad B(x_2, y_2) \quad \text{и} \quad C(x_3, y_3),$$
	не лежащие на одной прямой, являются последовательными вершинами параллелограмма. Найти координаты четвертой вершины.
}

\def \SimpleTwo {%
	\item Даны две точки $A(3, -2)$  и $B(1, 4)$. Точка $M$ лежит на прямой $AB$, причем $|AM| = 3|AB|$. Найдите координаты точки $M$, если
	\begin{enumerate}
		\item точка $M$ лежит по одну сторону от $A$ вместе с $B$;
		\item точки $M$ и $B$ лежат по разные стороны от $A$.
	\end{enumerate}
}


% Hard
\def \HardOne {%
	\item Даны три точки:
	$$A(x_1, y_1, z_1),\quad B(x_2, y_2, z_2)\quad  \text{и} \quad  C(x_3, y_3, z_3),$$
	не лежащие на одной прямой. Найдите координаты точки пересечения медиан~$\triangle ABC$.
}

\def \HardTwo {%
	\item Доказать, что если диагонали четырехугольника в точке пересечения делятся пополам, то этот четырехугольник --- параллелограмм.}


