\begin{enumerate}

\item Вычислить линейные комбинации:
    \[
        \begin{pmatrix}
            4 &  2 & 0 \\
            1 &  3 & 5 \\
            1 & -2 & 3
        \end{pmatrix}
        +2
        \begin{pmatrix}
           -2 &  1 & 0 \\
           -1 &  0 & 1 \\
            1 &  3 & 0
        \end{pmatrix}
        -4
        \begin{pmatrix}
           -1 &  0 & -1 \\
            0 &  1 &  2 \\
           -1 &  0 &  1
        \end{pmatrix}
        =
    \]
    \[
        2
        \begin{pmatrix}
            1 \\
            2 \\
            3 \\
            4
        \end{pmatrix}
        +
        \begin{pmatrix}
            4 \\
            3 \\
            2 \\
            1
        \end{pmatrix}
        -
        \begin{pmatrix}
           -2 \\
            6 \\
            7 \\
            6
        \end{pmatrix}
        =
    \]


\item Найти произведения матриц:
\begin{alignat*}{2}
    \begin{pmatrix}
        8 & 1 & 1 & 1
    \end{pmatrix}
    \cdot
    \begin{pmatrix}
        3 \\
        -1 \\
        -1 \\
        -5
    \end{pmatrix}
    &=
    & \hspace{3cm}
    \begin{pmatrix}
        8 \\
        1 \\
        1 \\
        1
    \end{pmatrix}
    \cdot
    \begin{pmatrix}
        1 & 2 & 1 & 2
    \end{pmatrix}
    &= \\[1em]
    \begin{pmatrix}
        1 & 2 & 3\\
        3 & 2 & 1
    \end{pmatrix}
    \cdot
    \begin{pmatrix}
        1 & 1 \\
        1 & 0 \\
        1 & 1
    \end{pmatrix}
    &=
    &
    \begin{pmatrix}
        1 & 0 & 0\\
        0 & 2 & 0\\
        0 & 0 & 10
    \end{pmatrix}
    ^{10}
    &=
\end{alignat*}


\item Найдите транспонированные матрицы:
\begin{alignat*}{2}
    \begin{pmatrix}
        1 & 2 & 3\\
       -2 & 1 & 2\\
       -3 & -2 & 1\\
    \end{pmatrix}
    ^T&=
    & \hspace{3cm}
    \begin{pmatrix}
        8 \\
        1 \\
        1 \\
        1 \\
    \end{pmatrix}
    ^T &=\hspace{3cm} \\[0.5em]
    \begin{pmatrix}
           \     &               &                       & \lambda_1  \\
           \     & \shiftright{-30pt}{\raisebox{20pt}{\bigzero}}      & \reflectbox{$\ddots$} &  \\
           \     & \lambda_{n-1} & \raisebox{-7pt}{\shiftright{17pt}{\bigzero}}              &  \\
       \lambda_n &               &                       & 
    \end{pmatrix}
    ^T&= 
\end{alignat*}


\item Вычислите детерминанты:
\begin{align*}
    \det 
    \begin{pmatrix}
        1 & 2 & 3 \\
        1 & 0 & 1\\
        1 & 1 & 1\\
    \end{pmatrix}
    &= 
    &
    \det 
    \begin{pmatrix}
        8000 & 1 & -1 \\
        -101 & 1 & 0 \\
        10 & 0 & 1 \\
    \end{pmatrix}
    &=
\end{align*}


\item Решите систему линейных уравнений, пользуясь правилом Крамера:
\begin{align*}
    \begin{cases}
        2x + y - z = 2 \\
        3x + y -2z = 3 \\
        x + z = 3
    \end{cases}
    & & 
    \begin{cases}
        x + 2y + 3z = 1\\
        x + y + z = 0\\
        y + z = 1
    \end{cases}
\end{align*}


\item Вычислите:
\begin{align*}
    A &= \begin{pmatrix}1 & 2 \\2 & 1 \end{pmatrix} &
    B &= \begin{pmatrix}2 & 1 \\1 & 2 \end{pmatrix} &
    (A^3 -4B) &=
\end{align*}


\item Используя свойства детерминанта, вычислите:
\[
    \det 
    \begin{pmatrix}
    0 & 1 & 2 & 3 & 4 & 5 \\
    1 & 2 & 3 & 4 & 5 & 6 \\
    1 & 2 & 3 & 5 & 7 & 9 \\
    1 & 2 & 4 & 6 & 8 & 10 \\
    1 & 2 & 3 & 4 & 5 & 7 \\
    1 & 2 & 3 & 4 & 6 & 8
    \end{pmatrix}
\]


\end{enumerate}